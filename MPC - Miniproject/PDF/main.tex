%preamble - package unclusion and set up
%page setup (page size, text size, page layout, chapters start on a new page).
%memoir is a form of book class that supports any kind of document.
\documentclass[fleqn,a4paper,11pt,twoside,openany]{memoir}

%setting the header and footer in that order:
\setheadfoot{28pt}{28pt} %if any problems are encountered, try changing the latter 28pt with 1cm.

%general package syntax: \usepackage[options]{package}

%setting language:
\RequirePackage[danish, english]{babel}

\usepackage{siunitx}

%this package makes it possible to treat any element as a float,
%figures and tables are by default treated as floats.
%read http://en.wikibooks.org/wiki/LaTeX/Floats,_Figures_and_Captions to specify your float.
\usepackage{float}
\usepackage{wrapfig}
\usepackage{placeins}

%this package makes it possible to make theorems and examples:
\usepackage{amsthm}
%setting the style of examples (parameters: plain, definition, remark):
%(definition is usually used for examples)
\theoremstyle{definition}
%the frist parameter is the syntax used in the document, the second is that which is printed in LaTex.
\newtheorem{example}{Eksempel}

%making it possible to use æ, ø and å:
\usepackage[utf8]{inputenc}
%helps with word division when using æ, ø and å, and makes it ps-font rather than bmp:
\usepackage[T1]{fontenc}

%package for implementation of graphic files:
\usepackage{graphicx}

%package for captions
\usepackage[nooneline]{caption}

%%package for implementation of math:
\usepackage{amsmath , amsfonts , amssymb, float}

%allowing use of color:
\usepackage{color}
%allowing use of more colors also in tables (see: http://en.wikibooks.org/wiki/LaTeX/Colors):
\usepackage[usenames,dvipsnames,svgnames,table]{xcolor}

%hyperlinks in the tabel of contents - comment this out before the report is printed.
\usepackage{hyperref}
\hypersetup{
	bookmarks = true,  % Show 'bookmark'-frame in pdf.
	colorlinks = true, % True = colored links, False = framed links.
	citecolor = blue,  % Link color for references.
	linkcolor = blue,  % Link color in table of contents.
	urlcolor = blue,   % Link color for extern URLs.
}

%makes it possible to refer to the name of a chapter rather than just the number.
\usepackage{nameref}

%package for writing program code in latex
\usepackage{listings}

\lstset{ 
language=C,               	 	% choose the language of the code
basicstyle=\footnotesize,       % the size of the fonts that are used for the code
numbers=left,                   % where to put the line-numbers
numberstyle=\footnotesize,      % the size of the fonts that are used for the line-numbers
stepnumber=1,                   % the step between two line-numbers. If it is 1 each line will be numbered
numbersep=5pt,                  % how far the line-numbers are from the code
backgroundcolor=\color{white},  % choose the background color. You must add \usepackage{color}
showspaces=false,               % show spaces adding particular underscores
showstringspaces=false,         % underline spaces within strings
showtabs=false,                 % show tabs within strings adding particular underscores
frame=single,           		% adds a frame around the code
tabsize=2,          			% sets default tabsize to 2 spaces
captionpos=b,           		% sets the caption-position to bottom
breaklines=true,       			% sets automatic line breaking
breakatwhitespace=false,    	% sets if automatic breaks should only happen at whitespace
escapeinside={\%*}{*)}          % if you want to add a comment within your code
}

%setting references (using numbers) and supporting i.a. Chicargo-style:
\usepackage{etex}
\usepackage{etoolbox}
\usepackage{keyval}
\usepackage{ifthen}
\usepackage{url}
\usepackage{csquotes}
\usepackage[backend=biber,url=true,doi=true,style=numeric,sorting=none]{biblatex}
\bibliography{myBib.bib}

%this package makes it possible include pdf pages in fx appendix;
%using  following syntax: \includepdf[pages={1}]{myfile.pdf}
\usepackage{pdfpages}

%%%MARGINER%%%
\setlrmarginsandblock{2cm}{2cm}{*}	% \setlrmarginsandblock{inner margin}{outer margin}{ratio}
\setulmarginsandblock{2cm}{2cm}{*}	% \setulmarginsandblock{top}{bottom}{ratio}
\checkandfixthelayout 			            % fixes stuff.

%Enables the use FiXme refferences. Syntax: \fixme{...}
%With "final" in stead of "draft" an error will ocure for every FiXme
%under compilation.
\usepackage[footnote,draft,english,silent,nomargin]{fixme}

%%%CHAPTERLAYOUT%%%
%setting the color of the chapter number
\definecolor{numbercolor}{gray}{0.7}
%Downloaded chapter-setup:
\newif\ifchapternonum
\makechapterstyle{jenor}{
  \renewcommand\printchaptername{}
  \renewcommand\printchapternum{}
  \renewcommand\printchapternonum{\chapternonumtrue}
  \renewcommand\chaptitlefont{\fontfamily{pbk}\fontseries{db}\fontshape{n}\fontsize{25}{35}\selectfont\raggedleft}
  \renewcommand\chapnumfont{\fontfamily{pbk}\fontseries{m}\fontshape{n}\fontsize{1in}{0in}\selectfont\color{numbercolor}}
  \renewcommand\printchaptertitle[1]{%
    \noindent
    \ifchapternonum
    \begin{tabularx}{\textwidth}{X}
    {\let\\\newline\chaptitlefont ##1\par} 
    \end{tabularx}
    \par\vskip-2.5mm\hrule
    \else
    \begin{tabularx}{\textwidth}{Xl}
    {\parbox[b]{\linewidth}{\chaptitlefont ##1}} & \raisebox{-15pt}{\chapnumfont \thechapter}
    \end{tabularx}
    \par\vskip2mm\hrule
    \fi
  }
}
%setting chapter style:
\chapterstyle{jenor}

%depth of numbered headlines (part/chapter/section/subsection):
\setsecnumdepth{none}
\maxsecnumdepth{none}
%depth of the table of contents:
\settocdepth{section}

% Makes sure LaTeX does not stretch the text at page break:
\raggedbottom

%macros - please read this file
%Macro for 'where'-enviroment was improved by Andrea and Niels :-)

%-----------UNITS-------------------------------------------
\newcommand{\unit}[1]{&& \left[\si{#1}\right]}
%
%\newcommand{\unit}[1]{[\si{#1}]}            %<<| Use these if you want equations to be
%\newcommand{\eq}[2]{&&\si{#1} &= \si{#2}&&} %<<| centered.. .. will appear scrambled
%                                            %  | from one equation to the next though..
%                                            %  | and does not work with long equations.. :/
%
%-----------------------------------------------------------

%-----------WHERE ENVIRONMENT-------------------------------
\newenvironment{where}{\leavevmode{\parindent=1em\indent} Where:\\}{}
\newcommand{\va}[3]
{
  \begin{tabular}{p{20pt} p{40pt} p{290pt} l}
    & { $#1$ } & { #2 } & \ifthenelse{\isempty{ #3 }}  {}  {[{\si{#3}}]} \\
  \end{tabular}\\
}
%-----------------------------------------------------------

%-----------TikZ SETTINGS-----------------------------------
\tikzset{
  block/.style    = {draw, thick, rectangle,
                     minimum height = 2.1em,
                     minimum width = 1.7em},
  sum/.style      = {draw, circle, inner sep=3pt} %<--Adder
}
%-----------------------------------------------------------

%------------VECTORS----------------------------------------
\renewcommand{\vec}[1]{\boldsymbol{\mathbf{#1}}} %TIP: If you are using TeXstudio you can open
                         %     the file by Ctrl+LeftClick on setup/macros.tex

\begin{document}
\section*{Model Predictive Control Exercise - Group 17gr832}
Alejandro Alonso García\\
Anders Egelund Kjeldal\\
Himal Kooverjee  \\
Niels Skov Vestergaard\\
Noelia Villarmarzo Arruñada
%
\subsection*{Description of the Problem}
The objective is to optimize the profit that can be obtained by running the plant showed in \autoref{fig:powerPlant}.

\begin{figure}[H]
    \centering
    \includegraphics[width=0.6\textwidth]{figures/powerPlant}
    \caption{Diagram of the plant, where $Q_\mathrm{G}$ is the power produced by the gas turbine, $Q_\mathrm{W}$ is the power produced by the waste burner, $Q_\mathrm{A,in}$ is the power fed into the accumulator, $Q_\mathrm{bp}$ is the power bypassing the accumulator, $Q_\mathrm{A,out}$ is the power leaving the accumulator, and $Q_\mathrm{E}$ is the power leaving the plant. The units of power are given in megawatt [MW]}
    \label{fig:powerPlant}
\end{figure}

The power flows in the plant have some constraints shown below:
%
\begin{flalign}
    Q_\mathrm{W} + Q_\mathrm{G} = Q_\mathrm{bp} + Q_\mathrm{A,in}, \ \ & \ \ Q_\mathrm{E} = Q_\mathrm{bp} + Q_\mathrm{A,out} \label{eq:equality} \\
    0 \leq Q_\mathrm{W} \leq 40, \ \  \ \  0 \leq Q_\mathrm{G} \leq 20 \ \ & \ \ 0, \leq Q_\mathrm{A,in} \leq 50, \ \  \ \ 0 \leq Q_\mathrm{A,out} \leq 25 \label{eq:inequality}
\end{flalign}

And the accumulator have some dynamics and some constraints as well
%
\begin{flalign}
    &E_\mathrm{A}[k+1] = E_\mathrm{A}[k] + (Q_\mathrm{A,in}[k] - Q_\mathrm{A,out}[k]) T_\mathrm{s} \label{eq:dynamics}\\
    &0 \leq E_\mathrm{A} \leq 200 \label{eq:inequality_E}
\end{flalign}

The profit can be optimized by scheduling the power production using the knowledge of the plant and  the future prices of gas, waste burning and electricity ($P_\mathrm{G}$, $P_\mathrm{W}$ and $P_\mathrm{E}$). The profit is given by
%
\begin{flalign}
    \sum_{i=k}^{k+L-1} (P_\mathrm{E}[i] Q_\mathrm{E}[i] - (P_\mathrm{G}[i] Q_\mathrm{G}[i] + P_\mathrm{W}[i] Q_\mathrm{W}[i])) T_\mathrm{s} \label{eq:cost}
\end{flalign}

\subsection*{Problem Formulation in CVX Form}
Once the problem has been described, the constraints, the dynamic equation and the cost function need to be rewritten in a suitable way to be able to optimize the problem using CVX.

The dynamic equation, \autoref{eq:dynamics}, of the accumulator can be rewritten by substituting backwards each $E_\mathrm{A}[k]$ until it depends only on the inputs and the initial state.
%
\begin{flalign}
    E_\mathrm{A}[k+1] = E_\mathrm{A}[k] + &(Q_\mathrm{A,in}[k] - Q_\mathrm{A,out}[k]) T_\mathrm{s} \\
    &... \nonumber \\
    E_\mathrm{A}[k+l] = E_\mathrm{A}[k] + &\sum_{i=k}^{k+l-1}(Q_\mathrm{A,in}[i] - Q_\mathrm{A,out}[i]) T_\mathrm{s} \\
    &... \nonumber \\
    E_\mathrm{A}[k+L] = E_\mathrm{A}[k] + &\sum_{i=k}^{k+L-1}(Q_\mathrm{A,in}[i] - Q_\mathrm{A,out}[i]) T_\mathrm{s}
\end{flalign}

This expression can also be rewritten in matrix form as follows
\begin{flalign}
    \begin{bmatrix}
        E_\mathrm{A}[k+1] \\
        E_\mathrm{A}[k+2] \\
        ...\\
        E_\mathrm{A}[k+L] \\
    \end{bmatrix}
    =  
    \begin{bmatrix}
        E_\mathrm{A}[k] \\
        E_\mathrm{A}[k] \\
        ...\\
        E_\mathrm{A}[k] \\
    \end{bmatrix}  
    +
    \begin{bmatrix}
        Q_\mathrm{A,in}[k] - Q_\mathrm{A,out}[k] \\
        Q_\mathrm{A,in}[k+1] - Q_\mathrm{A,out}[k+1] \\
        ... \\
        Q_\mathrm{A,in}[k+L-1] - Q_\mathrm{A,out}[k+L-1]
    \end{bmatrix}^\mathrm{T}
    \begin{bmatrix}
        1 & 1 & 1 & ... & 1 \\
        0 & 1 & 1 & ... & 1\\
        0 & 0 & 1 & ... & 1\\
        ... & ... & ... & ... & ... \\
        0 & 0 & 0 & ... & 1
    \end{bmatrix} \label{eq:matrix_dynamics}
\end{flalign}

Since there exists a constraint on the amount of charge that can be stored in the accumulator at each time, \autoref{eq:matrix_dynamics} can be used to write that constraint in an appropriate form.
\begin{flalign}
    0 \leq  
    \begin{bmatrix}
        E_\mathrm{A}[k] \\
        E_\mathrm{A}[k] \\
        ...\\
        E_\mathrm{A}[k] \\
    \end{bmatrix}  
    +
    \begin{bmatrix}
        Q_\mathrm{A,in}[k] - Q_\mathrm{A,out}[k] \\
        Q_\mathrm{A,in}[k+1] - Q_\mathrm{A,out}[k+1] \\
        ... \\
        Q_\mathrm{A,in}[k+L-1] - Q_\mathrm{A,out}[k+L-1]
    \end{bmatrix}^\mathrm{T}
    \begin{bmatrix}
        1 & 1 & 1 & ... & 1 \\
        0 & 1 & 1 & ... & 1\\
        0 & 0 & 1 & ... & 1\\
        ... & ... & ... & ... & ... \\
        0 & 0 & 0 & ... & 1
    \end{bmatrix}
    \leq 200 \label{eq:matrix_constraint}
\end{flalign}

The equality constraints in \autoref{eq:equality} can be used to rewrite the cost function, \autoref{eq:cost}, as a function of $Q_\mathrm{G}$, $Q_\mathrm{W}$, $Q_\mathrm{A,in}$ and $Q_\mathrm{A,out}$. This results in \autoref{eq:final_cost}
%
\begin{flalign}
    & \sum_{i=k}^{k+L-1} (P_\mathrm{E}[i] Q_\mathrm{E}[i] - (P_\mathrm{G}[i] Q_\mathrm{G}[i] + P_\mathrm{W}[i] Q_\mathrm{W}[i])) T_\mathrm{s} \\
    & \sum_{i=k}^{k+L-1} (P_\mathrm{E}[i] (Q_\mathrm{bp}[i] + Q_\mathrm{A,out}[i]) - (P_\mathrm{G}[i] Q_\mathrm{G}[i] + P_\mathrm{W}[i] Q_\mathrm{W}[i])) T_\mathrm{s} \\
    & \sum_{i=k}^{k+L-1} (P_\mathrm{E}[i] (Q_\mathrm{W}[i] + Q_\mathrm{G}[i] - Q_\mathrm{A,in}[i] + Q_\mathrm{A,out}) - (P_\mathrm{G}[i] Q_\mathrm{G}[i] + P_\mathrm{W}[i] Q_\mathrm{W}[i])) T_\mathrm{s} \\
    & \sum_{i=k}^{k+L-1} ((P_\mathrm{E}[i] - P_\mathrm{W}[i])Q_\mathrm{G}[i] + (P_\mathrm{E}[i] - P_\mathrm{G}[i])Q_\mathrm{G}[i] + P_\mathrm{E}[i] Q_\mathrm{A,out}[i] - P_\mathrm{E}[i] Q_\mathrm{A,in}[i]) T_\mathrm{s} \label{eq:final_cost}
\end{flalign}

This expression can also be written in matrix form to be suitable to use with CVX.
\footnotesize{
\begin{flalign}
    \begin{bmatrix}
        P_\mathrm{E}[k] - P_\mathrm{W}[k] \\
        P_\mathrm{E}[k+1] - P_\mathrm{W}[k+1] \\
        ...\\
        P_\mathrm{E}[k+L-1] -P_\mathrm{W}[k+L-1] \\
    \end{bmatrix}^\mathrm{T}  
    \begin{bmatrix}
        Q_\mathrm{W}[k] \\
        Q_\mathrm{W}[k+1] \\
        ...\\
        Q_\mathrm{W}[k+L-1] \\
    \end{bmatrix}  
    +
    \begin{bmatrix}
        P_\mathrm{E}[k] - P_\mathrm{G}[k] \\
        P_\mathrm{E}[k+1] - P_\mathrm{G}[k+1] \\
        ...\\
        P_\mathrm{E}[k+L-1] -P_\mathrm{G}[k+L-1] \\
    \end{bmatrix}^\mathrm{T}  
    \begin{bmatrix}
        Q_\mathrm{G}[k] \\
        Q_\mathrm{G}[k+1] \\
        ...\\
        Q_\mathrm{G}[k+L-1] \\
    \end{bmatrix}
    + \nonumber\\
    +
    \begin{bmatrix}
        P_\mathrm{E}[k] \\
        P_\mathrm{E}[k+1]\\
        ...\\
        P_\mathrm{E}[k+L-1]\\
     \end{bmatrix}^\mathrm{T}  
     \begin{bmatrix}
         Q_\mathrm{A_out}[k] \\
         Q_\mathrm{A_out}[k+1] \\
         ...\\
         Q_\mathrm{A_out}[k+L-1] \\
     \end{bmatrix}  
     -
     \begin{bmatrix}
         P_\mathrm{E}[k]\\
         P_\mathrm{E}[k+1]\\
         ...\\
         P_\mathrm{E}[k+L-1]\\
     \end{bmatrix}^\mathrm{T}  
     \begin{bmatrix}
         Q_\mathrm{A_in}[k] \\
         Q_\mathrm{A_in}[k+1] \\
         ...\\
         Q_\mathrm{A_in}[k+L-1] \\
     \end{bmatrix}\label{eq:matrix_cost}
\end{flalign}
}
\normalsize
\subsection*{Implementation}
The implementation of the optimization problem is shown in \autoref{cvxImplementation}. It includes the inequality constraints in \autoref{eq:inequality} and \ref{eq:matrix_constraint} as well as the cost defined in \autoref{eq:matrix_cost}.
%
\begin{lstlisting}[ language = Matlab,
					caption  = {Matlab code for the implementation of the problem with CVX},
					label    = cvxImplementation ]
cvx_begin quiet % The beginning of the optimization problem

    % Define the variables
    variable Q_W(L,1);
    variable Q_G(L,1);
    variable Q_A_in(L,1);
    variable Q_A_out(L,1);
    
    % Specify the optimization of cost
    minimize(-(P_E(k:k+L-1)'*Q_A_out+(P_E(k:k+L-1)'-P_G(k:k+L-1)')*Q_G+...
    (P_E(k:k+L-1)'-P_W(k:k+L-1)')*Q_W-P_E(k:k+L-1)'*Q_A_in)*Ts);
    
    % Constraints
    subject to 
    Q_W>=Q_W_min;
    Q_W<=Q_W_max;
    Q_G>=Q_G_min;
    Q_G<=Q_G_max;
    Q_A_in>=Q_A_in_min;
    Q_A_in<=Q_A_in_max;
    Q_A_out>=Q_A_out_min;
    Q_A_out<=Q_A_out_max;
    (Q_A_in-Q_A_out)'*triu(ones(L,L))>=E_A_min-E_A_sys(k);
    (Q_A_in-Q_A_out)'*triu(ones(L,L))<=E_A_max-E_A_sys(k);
    
cvx_end     % The end of the optimization problem
\end{lstlisting}

In each iteration of the loop it tries to minimize the cost along the horizon $L$ knowing the corresponding prices and the inequality constraints, by finding the optimal $Q_\mathrm{W}$, $Q_\mathrm{W}$, $Q_\mathrm{A,in}$ and $Q_\mathrm{A,out}$.

\subsection*{Results}
The results obtained are presented in \autoref{fig:E_A}, \ref{fig:Q_E}, \ref{fig:Q_W} and \ref{fig:Q_G}.

In \autoref{fig:E_A} the state of charge of the accumulator can be seen. It shall be noticed that the accumulator charges when the price of electricity is low and discharges when it is high. For this to be seen it is neccesary to look also at \autoref{fig:Q_E}, where the power production and price is depicted. Good examples appear at 90 hours and between 170 and 180 hours. The accumulator also charges when the power production of gas is cheaper, this is shown in \autoref{fig:Q_G}. The clearest example of this occurs between 110 and 120 hours.

From the power production seen in \autoref{fig:Q_E}, it can be said that the power produced is low when electricity is cheap and high otherwise. This is clear in the graph as when the price peaks up, so does the production and vice versa
\begin{figure}[H]
    \captionbox 
    {
    	State of charge in the accumulator.
        \label{fig:E_A}                                  
    }                                                                 
    {                                                                 
        \includegraphics[height=.37\textwidth]{figures/E_A}
    }                                                                    
    \hspace{5pt}                                                          
    \captionbox  
    {        
        Comparison of the power production and the price of electricity.    
        \label{fig:Q_E}                                    
    }                                                                     
    {   
        \includegraphics[height=.37\textwidth]{Q_E}            
    }                                                                             
\end{figure}
The information presented in \autoref{fig:Q_W} and \ref{fig:Q_G} shows how the electric power is produced. It can be observed how waste is used in almost all the simulation as its price is negative. The only exception occurs when the price of electricity is also negative. This leads to a stop in the electricity production and from gas and waste and a charging period of the accumulator using electric power from outside the plant. The use of gas is more interesting as it is used as long as its price is lower than that of electricity, else the production of electric power with gas is zero. Examples of this take place between 65 and 70 hours, between 90 and 95 hours and at 160 hours.
\begin{figure}[H]
    \captionbox
    {       
        Comparison of the power produced using waste and the price of waste burning.          
        \label{fig:Q_W}                                  
    }                                                                 
    {                                                                  
        \includegraphics[height=.37\textwidth]{figures/Q_W} 
    }                                                                    
    \hspace{5pt}                                                          
    \captionbox  
    {          
        Comparison of the power produced using gas and the difference between the price of electricity and the price of gas.
        \label{fig:Q_G}                                     
    }                                                                     
    {                                                                     
        \includegraphics[height=.37\textwidth]{Q_G}            
    }      
\end{figure}
All the information presented suggests that the plant operates taking into account the different prices and adjusting the power production in order to maximize the profit obtained when running.
\end{document}
