\section{Strategy 2: Leader-Follower}

The leader follower principle is used as a higher order principle of how to navigate a group of robots. The principle is that a leader is defined to lead the group of robots in an environment relative to a trajectory. Instead of all the individual robots track their respective trajectories, only the leader follows a trajectory and the following robots keep their position relative to the leader position.

The way that the followers maintain their position can be done in different ways e.g. potential fields \citep{pfmrm}, behavioural methods as Null Space Based behavioural methods \citep{arrichiello2006formation} or versions of a direct \ac{LOS} algorithm. With the focus in here the formation should be defined as a 'rigid' formation, where the follower's positions are defined as fixed distances from the leader. The term rigid is not a strict description in this case because the formation can vary a little all the time, but it is not flexible either \citep{976029}.

The positions should be defined individually for each of the followers to the position of the leader. This can be done through a formation constraint function, F($\eta$), which should be a strict convex function to ensure the formation constraint (the actual formation). This formation constraint function can be expressed in different ways, with the actual positions of the agents and the virtual leader or positions of the agents relative to the virtual leaders starting position.

From the decision table \ref{tab:decision-matrix} it can be seen that duckling formation and echelon formation have got a high rating. These are both branches of the leader-follower principle which often shows to be applicable in formation issues \citep{TKP04}, \citep{1013687}, \citep{976029}.

The duckling formation is a direct intercept algorithm, where the followers intercept their respective leaders and withholding a desired safety distance to their leaders. They will form a chain of leader-follower-follower-follower continuing with followers until the desired amount of robots are in the formation. This formation is named duckling since this will be as a family of ducks where the children follow directly after their parents and does not care about anything on their way.

The echelon formation is a branch from the duckling formation where, instead of a direct pursuit, the followers are given a offset from the leader thus spanning the formation. This offset can be given in different ways i.e. with a desired difference in distance in a fixed angle from the leader. In the same way as the duckling formation this also has the opportunity to expand with a desired amount of robots in the formation that follows their respective leaders.