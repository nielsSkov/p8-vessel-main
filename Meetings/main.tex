%implementing document formatting:
%page setup (page size, text size, page layout, chapters start on a new page).
%memoir is a form of book class that supports any kind of document.
\documentclass[fleqn,a4paper,11pt,twoside,openany]{memoir}

%setting the header and footer in that order:
\setheadfoot{28pt}{28pt} %if any problems are encountered, try changing the latter 28pt with 1cm.

%general package syntax: \usepackage[options]{package}

%setting language:
\RequirePackage[danish, english]{babel}

\usepackage{siunitx}

%this package makes it possible to treat any element as a float,
%figures and tables are by default treated as floats.
%read http://en.wikibooks.org/wiki/LaTeX/Floats,_Figures_and_Captions to specify your float.
\usepackage{float}
\usepackage{wrapfig}
\usepackage{placeins}

%this package makes it possible to make theorems and examples:
\usepackage{amsthm}
%setting the style of examples (parameters: plain, definition, remark):
%(definition is usually used for examples)
\theoremstyle{definition}
%the frist parameter is the syntax used in the document, the second is that which is printed in LaTex.
\newtheorem{example}{Eksempel}

%making it possible to use æ, ø and å:
\usepackage[utf8]{inputenc}
%helps with word division when using æ, ø and å, and makes it ps-font rather than bmp:
\usepackage[T1]{fontenc}

%package for implementation of graphic files:
\usepackage{graphicx}

%package for captions
\usepackage[nooneline]{caption}

%%package for implementation of math:
\usepackage{amsmath , amsfonts , amssymb, float}

%allowing use of color:
\usepackage{color}
%allowing use of more colors also in tables (see: http://en.wikibooks.org/wiki/LaTeX/Colors):
\usepackage[usenames,dvipsnames,svgnames,table]{xcolor}

%hyperlinks in the tabel of contents - comment this out before the report is printed.
\usepackage{hyperref}
\hypersetup{
	bookmarks = true,  % Show 'bookmark'-frame in pdf.
	colorlinks = true, % True = colored links, False = framed links.
	citecolor = blue,  % Link color for references.
	linkcolor = blue,  % Link color in table of contents.
	urlcolor = blue,   % Link color for extern URLs.
}

%makes it possible to refer to the name of a chapter rather than just the number.
\usepackage{nameref}

%package for writing program code in latex
\usepackage{listings}

\lstset{ 
language=C,               	 	% choose the language of the code
basicstyle=\footnotesize,       % the size of the fonts that are used for the code
numbers=left,                   % where to put the line-numbers
numberstyle=\footnotesize,      % the size of the fonts that are used for the line-numbers
stepnumber=1,                   % the step between two line-numbers. If it is 1 each line will be numbered
numbersep=5pt,                  % how far the line-numbers are from the code
backgroundcolor=\color{white},  % choose the background color. You must add \usepackage{color}
showspaces=false,               % show spaces adding particular underscores
showstringspaces=false,         % underline spaces within strings
showtabs=false,                 % show tabs within strings adding particular underscores
frame=single,           		% adds a frame around the code
tabsize=2,          			% sets default tabsize to 2 spaces
captionpos=b,           		% sets the caption-position to bottom
breaklines=true,       			% sets automatic line breaking
breakatwhitespace=false,    	% sets if automatic breaks should only happen at whitespace
escapeinside={\%*}{*)}          % if you want to add a comment within your code
}

%setting references (using numbers) and supporting i.a. Chicargo-style:
\usepackage{etex}
\usepackage{etoolbox}
\usepackage{keyval}
\usepackage{ifthen}
\usepackage{url}
\usepackage{csquotes}
\usepackage[backend=biber,url=true,doi=true,style=numeric,sorting=none]{biblatex}
\bibliography{myBib.bib}

%this package makes it possible include pdf pages in fx appendix;
%using  following syntax: \includepdf[pages={1}]{myfile.pdf}
\usepackage{pdfpages}

%%%MARGINER%%%
\setlrmarginsandblock{2cm}{2cm}{*}	% \setlrmarginsandblock{inner margin}{outer margin}{ratio}
\setulmarginsandblock{2cm}{2cm}{*}	% \setulmarginsandblock{top}{bottom}{ratio}
\checkandfixthelayout 			            % fixes stuff.

%Enables the use FiXme refferences. Syntax: \fixme{...}
%With "final" in stead of "draft" an error will ocure for every FiXme
%under compilation.
\usepackage[footnote,draft,english,silent,nomargin]{fixme}

%%%CHAPTERLAYOUT%%%
%setting the color of the chapter number
\definecolor{numbercolor}{gray}{0.7}
%Downloaded chapter-setup:
\newif\ifchapternonum
\makechapterstyle{jenor}{
  \renewcommand\printchaptername{}
  \renewcommand\printchapternum{}
  \renewcommand\printchapternonum{\chapternonumtrue}
  \renewcommand\chaptitlefont{\fontfamily{pbk}\fontseries{db}\fontshape{n}\fontsize{25}{35}\selectfont\raggedleft}
  \renewcommand\chapnumfont{\fontfamily{pbk}\fontseries{m}\fontshape{n}\fontsize{1in}{0in}\selectfont\color{numbercolor}}
  \renewcommand\printchaptertitle[1]{%
    \noindent
    \ifchapternonum
    \begin{tabularx}{\textwidth}{X}
    {\let\\\newline\chaptitlefont ##1\par} 
    \end{tabularx}
    \par\vskip-2.5mm\hrule
    \else
    \begin{tabularx}{\textwidth}{Xl}
    {\parbox[b]{\linewidth}{\chaptitlefont ##1}} & \raisebox{-15pt}{\chapnumfont \thechapter}
    \end{tabularx}
    \par\vskip2mm\hrule
    \fi
  }
}
%setting chapter style:
\chapterstyle{jenor}

%depth of numbered headlines (part/chapter/section/subsection):
\setsecnumdepth{none}
\maxsecnumdepth{none}
%depth of the table of contents:
\settocdepth{section}

% Makes sure LaTeX does not stretch the text at page break:
\raggedbottom
%Vectors
\renewcommand{\vec}[1]{\boldsymbol{\mathbf{#1}}}
\begin{document}
\renewcommand\chaptername{KAPITEL}
\renewcommand\contentsname{Indhold}
\renewcommand\figurename{Figur}
\renewcommand\tablename{Tabel}

\section*{Supervisor meeting\\ \small Wednesday, 8th of February of 2017}
\subsection{Project Plan}
\begin{itemize}
	\item Handle disturbances fits more to the semester requirements than trying to avoid obstacles.
	\item Following a route can be done in many ways, with way-points, trajectories or others. We should decide which to use.
\end{itemize}	
\subsection{Organization of the controllers in Low Level/High Level}
\begin{itemize}
	\item In order to decide if we keep it as it is or put the inner loop in the arduino, we should look into the dynamics and the time to process in the high level processor.
\end{itemize}
\subsection{Batteries and Power supplies}
\begin{itemize}
	\item The batteries should be the same, but we can use the power supply for now.
\end{itemize}
\subsection{Model}
\begin{itemize}
	\item We have a good simple starting model. It would be a good idea to keep it like this for now.
	\item We are not able to control the other three degrees of freedom but we would get coupling dynamics through the extra equations.
	\item The side-thrusters may not have any effect. Remove them if the goal is to sail and include them if it is to stay still.
	\item We can reuse the parameters obtained from previous reports.
	\item The simulation does not look correct now, it should be turning in a fixed circle at the end. 
	\item The control problem can be solved with the three degree model.
\end{itemize}
\subsection{Estimation}
\begin{itemize}
	\item What you do is fusing all your sensors in order to get overall better position and direction. For example, you get a bias with the IMU which is corrected by the GPS.
	\item For the estimation we need a six dimensional model in order to get a good estimation.
\end{itemize}
\subsection{GPS option}
\begin{itemize}
	\item We could use an RTK GPS, which will give us a cm precision. It is more expensive and it requires radio-connection to a base station.
\end{itemize}
\subsection{Test suggestion}
\begin{itemize}
	\item Drive in the parking lot and then check the position in the map. This could be used to test the estimation of the position.
\end{itemize}
\subsection{Miscellaneous}
\begin{itemize}
	\item We could start documenting the model.
	\item 'Guidance and Control of Ocean Vehicles', 'Marine Craft Hydrodynamics and Motion Control' and 'Marine Control Systems and ...' are books that we can look into
\end{itemize}
\subsection{Next Supervisor Meeting}
Tuesday, 14th of February at 9:00

\end{document}