%implementing document formatting:
%page setup (page size, text size, page layout, chapters start on a new page).
%memoir is a form of book class that supports any kind of document.
\documentclass[fleqn,a4paper,11pt,twoside,openany]{memoir}

%setting the header and footer in that order:
\setheadfoot{28pt}{28pt} %if any problems are encountered, try changing the latter 28pt with 1cm.

%general package syntax: \usepackage[options]{package}

%setting language:
\RequirePackage[danish, english]{babel}

\usepackage{siunitx}

%this package makes it possible to treat any element as a float,
%figures and tables are by default treated as floats.
%read http://en.wikibooks.org/wiki/LaTeX/Floats,_Figures_and_Captions to specify your float.
\usepackage{float}
\usepackage{wrapfig}
\usepackage{placeins}

%this package makes it possible to make theorems and examples:
\usepackage{amsthm}
%setting the style of examples (parameters: plain, definition, remark):
%(definition is usually used for examples)
\theoremstyle{definition}
%the frist parameter is the syntax used in the document, the second is that which is printed in LaTex.
\newtheorem{example}{Eksempel}

%making it possible to use æ, ø and å:
\usepackage[utf8]{inputenc}
%helps with word division when using æ, ø and å, and makes it ps-font rather than bmp:
\usepackage[T1]{fontenc}

%package for implementation of graphic files:
\usepackage{graphicx}

%package for captions
\usepackage[nooneline]{caption}

%%package for implementation of math:
\usepackage{amsmath , amsfonts , amssymb, float}

%allowing use of color:
\usepackage{color}
%allowing use of more colors also in tables (see: http://en.wikibooks.org/wiki/LaTeX/Colors):
\usepackage[usenames,dvipsnames,svgnames,table]{xcolor}

%hyperlinks in the tabel of contents - comment this out before the report is printed.
\usepackage{hyperref}
\hypersetup{
	bookmarks = true,  % Show 'bookmark'-frame in pdf.
	colorlinks = true, % True = colored links, False = framed links.
	citecolor = blue,  % Link color for references.
	linkcolor = blue,  % Link color in table of contents.
	urlcolor = blue,   % Link color for extern URLs.
}

%makes it possible to refer to the name of a chapter rather than just the number.
\usepackage{nameref}

%package for writing program code in latex
\usepackage{listings}

\lstset{ 
language=C,               	 	% choose the language of the code
basicstyle=\footnotesize,       % the size of the fonts that are used for the code
numbers=left,                   % where to put the line-numbers
numberstyle=\footnotesize,      % the size of the fonts that are used for the line-numbers
stepnumber=1,                   % the step between two line-numbers. If it is 1 each line will be numbered
numbersep=5pt,                  % how far the line-numbers are from the code
backgroundcolor=\color{white},  % choose the background color. You must add \usepackage{color}
showspaces=false,               % show spaces adding particular underscores
showstringspaces=false,         % underline spaces within strings
showtabs=false,                 % show tabs within strings adding particular underscores
frame=single,           		% adds a frame around the code
tabsize=2,          			% sets default tabsize to 2 spaces
captionpos=b,           		% sets the caption-position to bottom
breaklines=true,       			% sets automatic line breaking
breakatwhitespace=false,    	% sets if automatic breaks should only happen at whitespace
escapeinside={\%*}{*)}          % if you want to add a comment within your code
}

%setting references (using numbers) and supporting i.a. Chicargo-style:
\usepackage{etex}
\usepackage{etoolbox}
\usepackage{keyval}
\usepackage{ifthen}
\usepackage{url}
\usepackage{csquotes}
\usepackage[backend=biber,url=true,doi=true,style=numeric,sorting=none]{biblatex}
\bibliography{myBib.bib}

%this package makes it possible include pdf pages in fx appendix;
%using  following syntax: \includepdf[pages={1}]{myfile.pdf}
\usepackage{pdfpages}

%%%MARGINER%%%
\setlrmarginsandblock{2cm}{2cm}{*}	% \setlrmarginsandblock{inner margin}{outer margin}{ratio}
\setulmarginsandblock{2cm}{2cm}{*}	% \setulmarginsandblock{top}{bottom}{ratio}
\checkandfixthelayout 			            % fixes stuff.

%Enables the use FiXme refferences. Syntax: \fixme{...}
%With "final" in stead of "draft" an error will ocure for every FiXme
%under compilation.
\usepackage[footnote,draft,english,silent,nomargin]{fixme}

%%%CHAPTERLAYOUT%%%
%setting the color of the chapter number
\definecolor{numbercolor}{gray}{0.7}
%Downloaded chapter-setup:
\newif\ifchapternonum
\makechapterstyle{jenor}{
  \renewcommand\printchaptername{}
  \renewcommand\printchapternum{}
  \renewcommand\printchapternonum{\chapternonumtrue}
  \renewcommand\chaptitlefont{\fontfamily{pbk}\fontseries{db}\fontshape{n}\fontsize{25}{35}\selectfont\raggedleft}
  \renewcommand\chapnumfont{\fontfamily{pbk}\fontseries{m}\fontshape{n}\fontsize{1in}{0in}\selectfont\color{numbercolor}}
  \renewcommand\printchaptertitle[1]{%
    \noindent
    \ifchapternonum
    \begin{tabularx}{\textwidth}{X}
    {\let\\\newline\chaptitlefont ##1\par} 
    \end{tabularx}
    \par\vskip-2.5mm\hrule
    \else
    \begin{tabularx}{\textwidth}{Xl}
    {\parbox[b]{\linewidth}{\chaptitlefont ##1}} & \raisebox{-15pt}{\chapnumfont \thechapter}
    \end{tabularx}
    \par\vskip2mm\hrule
    \fi
  }
}
%setting chapter style:
\chapterstyle{jenor}

%depth of numbered headlines (part/chapter/section/subsection):
\setsecnumdepth{none}
\maxsecnumdepth{none}
%depth of the table of contents:
\settocdepth{section}

% Makes sure LaTeX does not stretch the text at page break:
\raggedbottom
%Vectors
\renewcommand{\vec}[1]{\boldsymbol{\mathbf{#1}}}
\begin{document}
\renewcommand\chaptername{KAPITEL}
\renewcommand\contentsname{Indhold}
\renewcommand\figurename{Figur}
\renewcommand\tablename{Tabel}

%\section*{Supervisor meeting\\ \small Tuesday, 14th of February of 2017}
\section*{Sensor Fusion Research}
\subsection{Wikipedia}
\begin{itemize}
	\item https://en.wikipedia.org/wiki/Extended\_Kalman\_filter
	\item https://en.wikipedia.org/wiki/Kalman\_filter
	\item In these two pages we can find the idea of what we need.
\end{itemize}
\subsection{GPS aided attitude and Heading Reference System Using MEMS Sensors}
\begin{itemize}
	\item Contains and explanation on how to use Kalman filter for a Micro Aerial Vehicle.
	\item It only estimates three states, but the idea of it can be extracted. 
	\item It includes a good model of the gyro and an incomplete one for the accelerometer.
	\item Weird GPS model.
\end{itemize}	
\subsection{Autonomous Navigation of a Small Boat using IMU/GPS/Compass}
\begin{itemize}
	\item It considers more states, but still only in a 2-D plane. Maybe we do not need to estimate every state in the system.
	\item It includes a good model for the accelerometer and the GPS.
	\item The explanation of the kalman filter is more or less what we need in theory, but it is quite messy.
	\item The principle is: Make a model of the sensors which includes bias and noise for the accelerometers and gyro and noise for the GPS. Then make a model of the system (State Space) and include the sensor biases in the model as new states in some way.   
\end{itemize}
\subsection{INS/GPS Fusion Architectures for Unmaned Aerial Vehicles}
\begin{itemize}
	\item Same gyro and acceleration model as those in the previous articles.
	\item Good magnetometer model, I think.
	\item It ends up with a 15 states Extended Kalman Filter. It is cumbersome to read, I don't know if we can get something out of the formula as it is there.
	\item The approach is again similar to the previous articles an it seems to be the way to go.
\end{itemize}
\subsection{Adaptive Kalman Filter Based Navigation: An IMU/GPS Integration Approach}
\begin{itemize}
	\item Simpler explanation of the principle, but it is to simple as it does not include biases for sensors.
\end{itemize}
\subsection{GPS/IMU Data Fusion using Multisensor Kalman Filtering}
\begin{itemize}
	\item The equations for the filter are presented in a more detail way, but models for the sensors are not included.
	\item Good for understanding how to set up the state/process model.
\end{itemize}
\subsection{Slides: Lecture Kalman Filter}
\begin{itemize}
	\item Simple explanation with a simple example. 
	\item Good for understanding why the biases are included as states.
	\item Example with a IMU and Gyro for estimating the attitude.
\end{itemize}
\subsection{A quaternion based Unscented Kalman Filter for Orientation tracking}
\begin{itemize}
	\item Detailed, long and heavy explanation of unscented kalman filter, which is suitable for working with the system in non-linear form. 
	\item The explanation uses quaternions in the paper.
\end{itemize}
\subsection{A new Extension of the Kalman Filter to Nonlinear Systems}
\begin{itemize}
	\item General explanation of the Unscented Kalman Filter, just in case we need to look into it.
	\item I do not think we should use this.
\end{itemize}
%\subsection{Next Supervisor Meeting}
%Wednesday, 22nd of February at 13:00

\end{document}