\chapter{Introduction}\label{chap:introduction}
The aim of this project is to automate an industrial robot to produce the characters of The Simpsons family using Lego Duplo bricks. The possible characters that are produced by the robot include Homer, Marge, Bart, Lisa and Maggie. Each character was made using a specific order of the bricks as given by:

\begin{itemize}
	\item Homer: blue, black and yellow
	\item Marge: green, yellow and blue
	\item Bart: blue, orange, yellow
	\item Lisa: yellow, orange, yellow
	\item Maggie: blue, yellow
\end{itemize}

Automation plays a key roll in production and manufacturing. Automating certain production lines may reduce costs, while improving product quality and reducing the production time. Industrial robots are generally used for repetitive tasks thus the motivation to automate the manufacturing of Lego Simpsons characters.

To fulfill these requirements the bricks located on the table need to be identified and classified by its color, location and orientation. These properties are extracted from an image taken using a camera placed directly above the bricks. Image processing techniques are used to classify the bricks and the orientation and location of the bricks are translated from the image's coordinate frame to the robot's coordinate frame.

To pick up the Lego bricks a specific gripper was designed such that the robot would grab the bricks with two opposite corners, thus ensuring the orientation of the brick in the gripper is precisely known.

The number of each character is determined and the robot is then given instructions to pick each block individually and create the characters on a green Lego grid.


This involves among other things:
\begin{itemize}
    \item Identifying which bricks are located on the table.
    \item Identifying the location of the bricks (e.g. the location of a black Dublo brick needed to build Homer)
    \item Determine the associated cost of each solution and the cheapest solutions.
    \item Grasping the bricks by means of a robot.
    \item Mounting the bricks on a plate or on top of other bricks
    \item Selecting the sequence in which you want to pick/place the bricks and build the figures.
\end{itemize}