\chapter{Preface}
This is a master's thesis concerning the platform named AAUSHIP. This
platform is an ongoing project at the Department of Automation and
Control.

The project have developed into the AAUSHIP platform, which is an
\ac{ASV} in working progress. In this thesis Jeppe Dam and Nick
Østergaard has been focusing on upgrading the hardware of the platform to
overcome implementation errors, designing an applicable model of the
platform and designing formation control strategies for a future fleet
of AAUSHIPs for surveying purposes. The idea of surveying have been in
focus for some time in the project, from where an initial
correspondence between \acf{AAU} and the Port of Aalborg has been
created. The Port of Aalborg has the interest in the AAUSHIP project
since they have an aim to become an intelligent harbour, and the work
of this project is a stepping stone for them. With the Port of Aalborg
in focus have the surveying purposes in the Limfjord become the area
of interest, which have given rise of directing the AAUSHIP to this
surveying purpose.

The work in the project have both given the project group an insight
of the hardware issues at the AAUSHIP, but have also given rise to
investigate the formation control aspects when dealing with groups of
\ac{ASV}. Since the early work on the AAUSHIP project only have been
on a single ship, the need for internal communication have not been
crucial. Now, to upgrade the platform, the ROS system have been
implemented both for future work but also for testing purposes behind
the desk. This have given the group the opportunity to simulate the
behaviour of the AAUSHIP in a controlled designed environment, and
afterwards implement and realize the design on the actual AAUSHIP.
\newpage

\subsection*{Thanks to}
\begin{description}
\item[Karl Damkjær Hansen] has been helpful with answering questions
	related to ROS, which have proven very useful when
	porting the original AAUSHIP platform operated through ROS. 
\item[The Port of Aalborg] has been open to share data, information and
	input in regard of surveying applications, and have shown interest in
	the project. As the state of the project is not fully working,
	and the AAUSHIP is a platform in progress, the direct corporation
	with them has not reached the implementation phase with their system
	yet.  
\end{description}


\begin{center}
  \begin{minipage}[t]{0.47\textwidth}
    \centering \vspace{1.5cm} \hrule \vspace{1mm} Nick \O stergaard
  \end{minipage}
  \hfill
  \begin{minipage}[t]{0.47\textwidth}
    \centering \vspace{1.5cm} \hrule \vspace{1mm} Jeppe Dam
  \end{minipage}
\end{center}


\newpage
\section*{Reading Guide}
The following report is divided into parts, related to different phases of the project. The parts are divided into chapters, the chapters describe different aspects of the project. The chapters are subdivided a number of times to further split up the content into specific topics. The report is ended with an appendix part, that contains all the material that is relevant to the project, but not necessarily interesting to the reader, such as measurement journals and transcripts of meetings.

\begin{description}
\item[Citations] in the report is done according to the Harvard method, the list of references can be found \vpageref{ch:litt}. The elements on the list of references are sorted by author.
\item[Acronyms] are written to their full extend, the first time they are used, with the acronym in parentheses, thereafter only the acronym is used. The list of acronyms can be found \vpageref{ch:acronyms}.
\item[Notation] of vectors are written in bold font with lower case letters ($\vec{v}$), matrices are written in bold font with upper case letters ($\vec{M}$). Single variables and constants are typeset in normal math ($x$).
\item[Additional] to the report is a website, which contains copies of web references and other digital files (source code, scripts and raw measurement data) that could be of interest to the reader.
\end{description}
